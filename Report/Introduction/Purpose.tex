\chapter{Purpose}

\section{The failure of the dictionary}
We find dictionary definitions to be insufficient, as they are either too restrictive, or too vague.

The Oxford English Dictionary \cite{oxfordDefinition} restrictively defines a sandwich as:

\begin{displayquote}
	An item of food consisting of two pieces of bread with a filling between them, eaten as a light meal.
\end{displayquote}

Whereas The Free Dictionary \cite{freeDefinition} more broadly defines a sandwich as:

\begin{displayquote}
	\begin{enumerate}
		\renewcommand{\theenumi}{\alph{enumi}}
		\item Two or more slices of bread with a filling such as meat or cheese placed between them.
		\item A partly split long or round roll containing a filling.
		\item One slice of bread covered with a filling.
	\end{enumerate}
\end{displayquote}

Since the Oxford English Dictionary definition requires two pieces of bread, this excludes sub sandwiches, which most would consider a sandwich.
This makes the Oxford definition too restrictive.

Also, both definitions fail to adequately define ``filling''.
The Free Dictionary gives meat and cheese as examples, but many people put lettuce and tomato in their sandwiches, neither of which are meat or cheese.

So, dictionary definitions of ``sandwich'' are insufficient to determine what a sandwich is.

\section{Legal background}
The question of what is a sandwich has been the centre of several legal publications.
We believe that these publications have failed to provide a strong definition of what a sandwich is, and they contradict each other.

For tax purposes, the New York State Department of Taxation and Finance \cite{taxDefinition} says:
\begin{displayquote}
	\textit{Sandwiches} include cold and hot sandwiches of every kind that are prepared and ready to be eaten, whether made on bread, on bagels, on rolls, in pitas, in wraps, or otherwise, and regardless of the filling or number of layers. A sandwich can be as simple as a buttered bagel or roll, or as elaborate as a six-foot, toasted submarine sandwich.

	Some examples of taxable sandwiches include:
	\begin{itemize}
		\item common sandwiches, such as:
		\begin{itemize}
			\item BLTs (bacon, lettuce, and tomato sandwiches);
			\item club sandwiches;
			\item cold cut sandwiches;
			\item grilled cheese sandwiches;
			\item peanut butter and jelly sandwiches
			\item salad-type sandwiches (e.g., chicken, egg, ham, and tuna);
		\end{itemize}
		\item bagel sandwiches (served buttered or with spreads, or otherwise as a sandwich);
		\item burritos
		\item cheese-steak sandwiches;
		\item croissant sandwiches;
		\item fish fry sandwiches;
		\item flatbread sandwiches;
		\item breakfast sandwiches;
		\item gyros;
		\item hamburgers on buns, rolls, etc.;
		\item heroes, hoagies, torpedoes, grinders, submarines, and other such sandwiches;
		\item hot dogs and sausages on buns, rolls, etc.;
		\item melt sandwiches;
		\item open-faced sandwiches;
		\item panini sandwiches;
		\item Reuben sandwiches; and
		\item wraps and pita sandwiches.
	\end{itemize}
\end{displayquote}

This is a very broad definition, but it is also quite comprehensive and informative.
It is important to note that \cite{taxDefinition} defines burritos as sandwiches.
However, other legal cases contradict this definition.

A case in the Commonwealth of Massachusetts Superior Court entitled White City Shopping Center, LP v. PR Restaurants, LLC dba Bread Panera \cite{whiteCityCase} involved two companies in a dispute over whether or not burritos are sandwiches.
In this case, the court ruled that burritos are not sandwiches.
This contradicts the definition in New York State tax law.
So, we can clearly see that there is no legal consensus on this matter.

Furthermore, the legal scholar Marjorie Florestal argues that the decision of the White City case is rooted in classist and racial views of sandwich cuisine \cite{classPaper}:
\begin{displayquote}
	The burrito meets resistance not just because of its class but also because of its race---and the way the two play off each other.
\end{displayquote}

So, we have established that the definition of a sandwich is inconclusive among both the linguist and legal communities \cite{oxfordDefinition, freeDefinition, taxDefinition, whiteCityCase}.
The question is also of importance for better understanding class systems and race in our society \cite{classPaper}.
