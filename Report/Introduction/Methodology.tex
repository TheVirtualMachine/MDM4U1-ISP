\chapter{Methodology}

\section{Survey type}
We conducted our survey as mix of a stratified, voluntary, and random sample.
We surveyed roughly 10\% of the William Lyon Mackenzie population.
With a sample size this large, most bias should be eliminated.

We aimed for a stratified sample of academic streams, as academic streams were related to our second hypothesis.
Unfortunately, we failed to collect a perfectly stratified sample of streams.
However, we did perform an analysis of the data and found that to be inconsequential.

According to the William Lyon Mackenzie student services department, roughly 26.67\% of the students at the school are in the MaCS program.
10.37\% are in the Gifted program.
62.96\% are in some other program.
There are roughly 1350 students at the school.
We could not find information on the number of teachers, but we estimate it to be between 60 and 70.

We made manual changes to categorical data to correct for similar, blank, and inappropriate responses.
As part of this, we grouped ethnicity into the following 11 categories: Caucasian, Chinese, East Asian, Filipino, Jewish, Korean, Middle Eastern, Mixed, Other, South Asian, and Vietnamese.

\subsection*{Demographic questions}
For demographic information, we asked participants for their grade (with teacher as an option), favourite subjects, and ethnic background.
We asked students for their academic stream, and teachers for their department.

Since the number of teachers surveyed was small, we do not do any analysis on teachers departments, and instead treat them as a separate grade and academic stream.
We also do not analyze the data on favourite subjects since it is very noisy (see \vref{fig:noisySubjects}).

\begin{figure}[H]
	\caption{A bar plot of subjects, showing the noisiness and uselessness of the subject data.}
	\label{fig:noisySubjects}
	\graphimage[width=\textwidth]{SubjectsBarPlot}
\end{figure}

\subsection*{Food questions}
We asked respondents 43 questions related to sandwiches and their ingredients.
We use all of this data.
