\chapter{Outliers}\label{chap:outliers}

We begin our analysis by plotting the purity vs orthodoxy and performing a linear, quadratic, and locally weighted analysis in \vref{fig:purityOrthodoxyStreamOutliers}.

The solid red curve is the result of the locally weighted analysis, and the shaded area is the 99\% confidence interval for that analysis.
The dashed black line is the result of the linear line of best fit.
The dotted blue line is the result of the quadratic curve of best fit.

\section{With outliers}
\begin{figure}[H]
	\graphimage{StreamPurityVsOrthodoxyWithOutliers}
	\caption{A plot of purity vs orthodoxy with data points coloured based on stream.}
	\label{fig:purityOrthodoxyStreamOutliers}
\end{figure}

The linear regression has a positive slope, which proves the first hypothesis that purity and orthodoxy are positively correlated.
We will investigate the strength and significance of the correlation in \vref{chap:significance}.

\section{Identifying outliers}
From the this analysis, it is clear that some points are extremely deviate.
We next create a residual plot using the quadratic model in \vref{fig:quadraticResidualOutliers}.

\begin{figure}[H]
	\graphimage{QuadraticResidualPlotWithOutliers}
	\caption{A residual plot using the quadratic regression analysis method.}
	\label{fig:quadraticResidualOutliers}
\end{figure}

We calculate the interquartile range of the residuals and remove any data that are $1.5$ times the ICQ above the third quartile or $1.5$ times the ICQ below the first quartile.
This is done using the program in \vref{src:outliers}.
This program then creates a new file for analyzing with the following respondents removed: 7, 51, 59, 62, 64, 69, 81, 82, 83, 103, 113, 120, 125, and 130.

\section{Outliers removed}
We once again plot orthodoxy vs purity by stream in \vref{fig:purityOrthodoxyStreamNoOutliers}, this time with the outliers removed.
\begin{figure}[H]
	\graphimage{StreamPurityVsOrthodoxyNoOutliers}
	\caption{A plot of purity vs orthodoxy with data points coloured based on stream. Outliers have been removed.}
	\label{fig:purityOrthodoxyStreamNoOutliers}
\end{figure}
The solid red curve is the result of the locally weighted analysis, and the shaded area is the 99\% confidence interval for that analysis.
The dashed black line is the result of the linear line of best fit.
The dotted blue line is the result of the quadratic curve of best fit.

\section{Effect of outliers on the data}
Removing the outliers does not significantly effect the curves due to the high sample size.
However, it does greatly increase the accuracy of the model and only a handful of points were removed.
