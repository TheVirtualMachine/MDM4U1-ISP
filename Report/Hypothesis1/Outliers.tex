\chapter{Outliers}\label{chap:outliers}

We begin our analysis by plotting the purity vs orthodoxy and performing a linear, quadratic, and locally weighted analysis in \vref{fig:purityOrthodoxyStreamOutliers}.

The solid red curve is the result of the locally weighted analysis, and the shaded area is the 99\% confidence interval for that analysis.
The dashed black line is the result of the linear line of best fit.
The dotted blue line is the result of the quadratic curve of best fit.

\section{With outliers}
\begin{figure}[H]
	\graphimage{StreamPurityVsOrthodoxyWithOutliers}
	\caption{A plot of purity vs orthodoxy with data points coloured based on stream.}
	\label{fig:purityOrthodoxyStreamOutliers}
\end{figure}

The linear regression has a positive slope, which proves the first hypothesis that purity and orthodoxy are positively correlated.
We will investigate the significance of the correlation in \vref{chap:significance}.

\subsection{Correlation calculations including outliers}
We use the program in \vref{src:significance} to calculate the numbers in this subsection.

\subsubsection{Linear model}
\paragraph{Equation of the linear model}
\[y \approx \num{0.1009528}x + \num{0.6800947}\]
\paragraph{$r$ values of the linear model}
\[r \approx \num{0.125253454074209}\]
\[r^2 \approx \num{0.01568842775752}\]

\subsubsection{Quadratic model}
\paragraph{Equation of the quadratic model}
\[y \approx \num{-1.3025366}x^2 + \num{0.3363544}x + \num{0.6757950}\]
\paragraph{$r^2$ value of the quadratic model}
\[r^2 \approx \num{0.25095758979336}\]

\section{Identifying outliers}
From this analysis, it is clear that some respondents are highly deviant.
We next use the quadratic mode to create a residual plot in \vref{fig:quadraticResidualOutliers}.

\begin{figure}[H]
	\graphimage{QuadraticResidualPlotWithOutliers}
	\caption{A residual plot using the quadratic regression analysis method.}
	\label{fig:quadraticResidualOutliers}
\end{figure}

We calculate the interquartile range of the residuals and remove any data that are $1.5$ times the ICQ above the third quartile or $1.5$ times the ICQ below the first quartile.
This is done using the program in \vref{src:outliers}.
This program then creates a new file for analyzing with the following respondents removed: 7, 51, 59, 62, 64, 69, 81, 82, 83, 103, 113, 120, 125, and 130.

\section{Outliers removed}
We once again plot orthodoxy vs purity by stream in \vref{fig:purityOrthodoxyStreamNoOutliers}, this time with the outliers removed.
\begin{figure}[H]
	\graphimage{StreamPurityVsOrthodoxyNoOutliers}
	\caption{A plot of purity vs orthodoxy with data points coloured based on stream. Outliers have been removed.}
	\label{fig:purityOrthodoxyStreamNoOutliers}
\end{figure}
The solid red curve is the result of the locally weighted analysis, and the shaded area is the 99\% confidence interval for that analysis.
The dashed black line is the result of the linear line of best fit.
The dotted blue line is the result of the quadratic curve of best fit.

\subsection{Correlation calculations with outliers removed}
We use the program in \vref{src:significance} to calculate the numbers in this subsection.

\subsubsection{Linear model}
\paragraph{Equation of the linear model}
\[y \approx \num{0.1138999}x + \num{0.7360640}\]
\paragraph{$r$ values of the linear model}
\[r \approx \num{0.209580523324373}\]
\[r^2 \approx \num{0.043923995756918}\]

\subsubsection{Quadratic model}
\paragraph{Equation of the quadratic model}
\[y \approx \num{-1.3512289}x^2 + \num{0.3633135}x + \num{0.7303165}\]
\paragraph{$r^2$ value of the quadratic model}
\[r^2 \approx \num{0.651494497892727}\]

\section{Effect of outliers on the data}
Removing the outliers does not significantly effect the curves due to the high sample size.
However, it does greatly increase the accuracy of the model and only a handful of points were removed.
This is especially notable in the $r^2$ value for the quadratic model, which more than doubles.

Unless otherwise specified, the rest of this paper will be excluding outliers from calculations.
