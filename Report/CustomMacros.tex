% Command for units with variables.
\NewDocumentCommand{\varSI}{}{\SI[number-math-rm=\mathnormal,parse-numbers=false]}

% Linear selection
\NewDocumentCommand{\linearselect}{}{%
	\begin{center}
		\begin{tabular}{ccccccccccc}
			\textbigcircle & \textbigcircle &\textbigcircle &\textbigcircle &\textbigcircle &\textbigcircle &\textbigcircle &\textbigcircle &\textbigcircle &\textbigcircle &\textbigcircle\\
			0 & 1 & 2 & 3 & 4 & 5 & 6 & 7 & 8 & 9 & 10
		\end{tabular}
	\end{center}
}

% Question commands.
\NewDocumentCommand{\multiplechoicequestion}{m o}{\subsection{#1}\IfNoValueF{#2}{#2~\\}\textit{Respondents were asked to check all that apply.}}
\NewDocumentCommand{\singlechoicequestion}{m o}{\subsection{#1}\IfNoValueF{#2}{#2~\\}\textit{Respondents were asked to select only one option.}}
\NewDocumentCommand{\shortanswerquestion}{m o}{\subsection{#1}\IfNoValueF{#2}{#2~\\}\textit{Respondents were asked to provide a free response.}\\\\\blank[0.8\textwidth]}
\NewDocumentCommand{\sandwichquestion}{m}{\subsection{#1}\textit{Respondents were asked to select only one option, with 0 being ``Not at all a sandwich'' and 10 being ``Definitely a sandwich''.}\linearselect}
\NewDocumentCommand{\ingredientquestion}{m}{\subsection{#1}\textit{Respondents were asked to select only one option, with 0 being ``Not at all a sandwich ingredient'' and 10 being ``Definitely a sandwich ingredient''.}\linearselect}
\NewDocumentCommand{\surveyimage}{m}{\begin{center}\includegraphics[height=1in]{#1}\end{center}}
\NewDocumentCommand{\graphimage}{O{} m}{\begin{center}\includegraphics[#1]{#2}\end{center}}

\NewTasks[style=multiplechoice]{multiplechoice}[\choice](1)
\NewTasks[label=\textbigcircle]{singlechoice}[\choice](1)

\NewDocumentCommand{\blank}{O{3em}}{\rule{#1}{0.5pt}}

% Math macros.
\newcommand{\real}{\mathbb{R}}
\newcommand{\such}{\mid}
\newcommand{\from}{:}
\newcommand{\integer}{\mathbb{Z}}
\newcommand{\nat}{\mathbb{N}}
\newcommand{\complex}{\mathbb{C}}
\DeclarePairedDelimiter{\ceil}{\lceil}{\rceil}
\DeclarePairedDelimiter{\floor}{\lfloor}{\rfloor}

% Environment for definition of theorems.
\theoremstyle{plain}
\newtheorem{theorem}{Theorem}

% Environment for miscellaneous definitions.
\theoremstyle{definition}
\newtheorem{definition}{Definition}
