\documentclass[letterpaper,12pt]{report}

% Fonts and typography
\usepackage[utf8]{inputenc}
\usepackage{lmodern}
\usepackage{microtype}
\usepackage{parskip}
\usepackage{titling}

% Math
\usepackage{amsmath}
\usepackage{amssymb}
\usepackage{amsthm}
\usepackage{commath}
\usepackage{siunitx}
\usepackage{mathtools}
\usepackage{esvect}

% Miscellaneous
\usepackage{xparse}
\usepackage{import}
\usepackage{float}
\usepackage{graphicx}
%\usepackage[draft]{graphicx}
\usepackage{textcomp}
\usepackage{tasks}
\usepackage{listings}
\usepackage{float}
\usepackage{longtable}

% Citations
\usepackage[backend=biber,urldate=iso8601]{biblatex}
\usepackage[hidelinks]{hyperref}

% References
\usepackage{varioref}

\addbibresource{References.bib}
\setlength{\bibitemsep}{\baselineskip}

% Title page setup.
\newcommand{\subtitle}[1]{\gdef\subT{#1}}
\newcommand{\subT}{}

\newcommand{\institute}[1]{\gdef\instuteName{#1}}
\newcommand{\instuteName}{}

% Title page formatting.
\pretitle{\begin{center}\Huge}
\posttitle{\par\end{center}\vskip 1ex}
\renewcommand{\maketitlehookb}{\begin{center}\LARGE\subT\end{center}\vskip 10ex}
\renewcommand{\maketitlehookc}{\begin{center}\Large\instuteName\end{center}\vskip 15ex}

% Reference formatting.
\labelformat{chapter}{chapter~#1}
\labelformat{section}{section~#1}
\labelformat{subsection}{subsection~#1}
\labelformat{figure}{figure~#1}
\labelformat{table}{table~#1}
\labelformat{equation}{equation~(#1)}

% Source code formatting.
\lstset{ %
	basicstyle=\ttfamily\footnotesize, % the size of the fonts that are used for the code
	breakatwhitespace=false,           % sets if automatic breaks should only happen at whitespace
	breaklines=true,                   % sets automatic line breaking
	captionpos=b,                      % sets the caption-position to bottom
	deletekeywords={...},              % if you want to delete keywords from the given language
	extendedchars=true,                % lets you use non-ASCII characters; for 8-bits encodings only, does not work with UTF-8
	frame=single,	                   % adds a frame around the code
	keepspaces=true,                   % keeps spaces in text, useful for keeping indentation of code (possibly needs columns=flexible)
	numbers=left,                      % where to put the line-numbers; possible values are (none, left, right)
	numbersep=5pt,                     % how far the line-numbers are from the code
	showspaces=false,                  % show spaces everywhere adding particular underscores; it overrides 'showstringspaces'
	showstringspaces=false,            % underline spaces within strings only
	showtabs=false,                    % show tabs within strings adding particular underscores
	tabsize=8,	                   % sets default tabsize to 8 spaces
}

% Image formatting.
\graphicspath{ {./Images/} }

\frenchspacing

% Command for units with variables.
\NewDocumentCommand{\varSI}{}{\SI[number-math-rm=\mathnormal,parse-numbers=false]}

% Linear selection
\NewDocumentCommand{\linearselect}{}{%
	\begin{center}
		\begin{tabular}{ccccccccccc}
			\textbigcircle & \textbigcircle &\textbigcircle &\textbigcircle &\textbigcircle &\textbigcircle &\textbigcircle &\textbigcircle &\textbigcircle &\textbigcircle &\textbigcircle\\
			0 & 1 & 2 & 3 & 4 & 5 & 6 & 7 & 8 & 9 & 10
		\end{tabular}
	\end{center}
}

% Question commands.
\NewDocumentCommand{\multiplechoicequestion}{m o}{\subsection{#1}\IfNoValueF{#2}{#2~\\}\textit{Respondents were asked to check all that apply.}}
\NewDocumentCommand{\singlechoicequestion}{m o}{\subsection{#1}\IfNoValueF{#2}{#2~\\}\textit{Respondents were asked to select only one option.}}
\NewDocumentCommand{\shortanswerquestion}{m o}{\subsection{#1}\IfNoValueF{#2}{#2~\\}\textit{Respondents were asked to provide a free response.}\\\\\blank[0.8\textwidth]}
\NewDocumentCommand{\sandwichquestion}{m}{\subsection{#1}\textit{Respondents were asked to select only one option, with 0 being ``Not at all a sandwich'' and 10 being ``Definitely a sandwich''.}\linearselect}
\NewDocumentCommand{\ingredientquestion}{m}{\subsection{#1}\textit{Respondents were asked to select only one option, with 0 being ``Not at all a sandwich ingredient'' and 10 being ``Definitely a sandwich ingredient''.}\linearselect}
\NewDocumentCommand{\surveyimage}{m}{\begin{center}\includegraphics[height=1in]{#1}\end{center}}
\NewDocumentCommand{\graphimage}{O{} m}{\begin{center}\includegraphics[#1]{#2}\end{center}}

\NewTasks[style=multiplechoice]{multiplechoice}[\choice](1)
\NewTasks[label=\textbigcircle]{singlechoice}[\choice](1)

\NewDocumentCommand{\blank}{O{3em}}{\rule{#1}{0.5pt}}

% Math macros.
\newcommand{\real}{\mathbb{R}}
\newcommand{\such}{\mid}
\newcommand{\from}{:}
\newcommand{\integer}{\mathbb{Z}}
\newcommand{\nat}{\mathbb{N}}
\newcommand{\complex}{\mathbb{C}}
\DeclarePairedDelimiter{\ceil}{\lceil}{\rceil}
\DeclarePairedDelimiter{\floor}{\lfloor}{\rfloor}

% Environment for definition of theorems.
\theoremstyle{plain}
\newtheorem{theorem}{Theorem}

% Environment for miscellaneous definitions.
\theoremstyle{definition}
\newtheorem{definition}{Definition}

