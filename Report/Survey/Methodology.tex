\chapter{Methodology}

\section{Axes metrics}
We have two metrics used in our calculations: purity and orthodoxy.

Purity is how pure a respondents definition of a sandwich is.
The less things a respondent considers a sandwich, the greater their purity score will be.
Similarly, the more things a respondent considers a sandwich, the lower their purity score will be.

Orthodoxy is a measure of how much a respondent differs from the mean set of responses.
The less a respondent's answers differ from the mean set of answers, the greater their orthodoxy score will be.
Similarly, the more a respondent's answers differ from the mean set of answers, the lower their orthodoxy score will be.

Both purity and orthodoxy are bound in the range $[-1, 1]$.

We describe the general scoring system in \cref{subsec:scoring}.
This scoring system is used to calculate the purity metric described in \cref{subsec:purityMetric} and the orthodoxy metric described in \cref{subsec:orthodoxyMetric}.

\subsection{Scoring}\label{subsec:scoring}
While participants answered each sandwich question on a 0 to 10 scale, it is more convenient to perform calculations using a $-5$ to 5 scale.
We converted responses from the 0 to 10 scale to the $-5$ to 5 scale by subtracting each response from 5.

Formally, for each response to a sandwich question, we calculate the score for the response by the passing the response through the sandwich spectrum function, defined in \cref{def:sandwichSpectrum}

\begin{definition}[Sandwich spectrum function]\label{def:sandwichSpectrum}
	The sandwich spectrum function is defined as:
	\begin{equation}
		s \from \set{x \in \integer \such 0 \leq x \leq 10} \to \set{x \in \integer \such -5 \leq x \leq 5}\ \text{by}\ s(x) = 5 - x
	\end{equation}
\end{definition}

We can create a table for $s$:
\begin{equation*}
	\begin{array}{l|r}
		x  & s(x)\\\hline
		0  &  5\\
		1  &  4\\
		2  &  3\\
		3  &  2\\
		4  &  1\\
		5  &  0\\
		6  & -1\\
		7  & -2\\
		8  & -3\\
		9  & -4\\
		10 & -5\\
	\end{array}
\end{equation*}

One will note that this gives responses that were originally high a lower score.
This is intentional.
\Cref{subsec:purityMetric} will show it to be useful for calculating the purity metric.
\Cref{subsec:orthodoxyMetric} will show it to be irrelevant for calculating the orthodoxy metric.

\subsection{Purity}\label{subsec:purityMetric}

\subsection{Orthodoxy}\label{subsec:orthodoxyMetric}
